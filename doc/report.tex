\documentclass[letterpaper, 12pt, titlepage]{article}
\usepackage[utf8]{inputenc}
\usepackage[english]{babel}
\usepackage[margin=1in]{geometry}
\usepackage[]{parskip}
\usepackage{graphicx}
\usepackage{paralist} % compactitem
\usepackage{multirow}
\usepackage{longtable}
\usepackage{float}

% Custom commands
\newcommand{\figRef}[1]{Figure \ref{#1}}
\newcommand{\tabRef}[1]{Table \ref{#1}}
\newcommand{\eqRef}[1]{(\ref{#1})}

\title{Compiler Project Report}
\author{
	Wayne Stegner
}
\date{
	\today
}

\begin{document}
	\maketitle
	\par \textbf{Code Structure}
	\par My compiler is written in C++ with an object-oriented approach.
	It contains the scanner, parser, and type checker for now, but I intend to
	complete code generation and runtime this Summer.
	The \texttt{Parser} class facilitates the overall flow of the
	recursive-descent parse.
	It interacts with the \texttt{Scanner} class to grab one token at a time,
	only ever looking at the current token.
	While the \texttt{Scanner} produces the tokens, it does not add them to the
	symbol table because it lacks semantic information such as scope and whether
	it should lookup or insert the symbol.
	Instead, the \texttt{Parser} inserts/looks up symbols through the
	\texttt{Environment} at appropriate places in the parse tree.
	\par The \texttt{Environment} abstracts scoping from the symbol table except
	for the case of inserting a symbol, where the \texttt{Parser} passes a
	\texttt{bool} value to distinguish global vs local inserts.
	The \texttt{Environment} class holds a global \texttt{SymbolTable} object and
	a local stack of \texttt{SymbolTable} objects to accommodate the scoping
	requirements.
	The \texttt{SymbolTable} class is mostly a wrapper to handle lookups and
	insertions to a STL hash map with string keys and token values.

	\par Tokens are implemented in multiple types using polymorphism, and they also
	serve as the symbol table entries.
	The base \texttt{Token} class stores a \texttt{TokenType} enumeration to help
	the parser identify grammar elements, a string denoting the token value, and a
	\texttt{TypeMark} enumeration for data type information.
	The \texttt{IdToken} class is used for identifiers, both variables and
	procedures.
	This class inherits from \texttt{Token}, adding an integer for counting
	elements, a boolean for denoting procedures, and a vector of \texttt{IdToken}s
	for holding parameter types.
	The \texttt{LiteralToken} template class is used to allow different data
	type values to be returned from the \texttt{Scanner}.
	Each token is returned from the scanner as a base \texttt{Token} pointer, and
	they are dynamically cast to child classes as needed in the \texttt{Parser}.

	\par The \texttt{TypeChecker} class holds functions useful for checking
	compatibilities of data types and array sizes for various operators.
	The functions are invoked during the parser, but the functions abstract some
	semantics regarding type matching (e.g.\ floats and integers are compatible
	for addition, but not a bitwise operator).
	The array size checking ensures that the arrays are either the same size or
	one of the operands is a scalar value.

	\par For error recovery, the parser takes various actions depending on the
	error.
	Type mismatches or size mismatches generally just require reporting the error
	and then continuing the parse as normal.
	However, the case of an unexpected token typically results in a panic mode
	recovery.
	Recovery mode uses \texttt{;} and \texttt{EOF} as sync tokens, and when a
	panic happens in either a \texttt{<declaration>} or a \texttt{<statement>},
	the current production is abandoned.
	I plan to make error recovery a bit more intelligent as I finish the compiler
	this Summer.

	\par \textbf{Running the Compiler}
	\par Prerequisite: \texttt{make} and \texttt{gcc} are installed.
	Run \texttt{make} in the project root directory, producing the executable
	\texttt{./bin/compiler}.
	Run \texttt{./bin/compiler -h} for usage.
	Example: \texttt{./bin/compiler -l 1 -i code.src -l log.txt} compiles
	\texttt{./code.src} with log level info and saves the full debug log to
	\texttt{./log.txt}.
	No binary is produced (yet).
\end{document}
